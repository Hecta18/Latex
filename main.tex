\documentclass[12pt, es]{article}
\usepackage{graphicx} % Required for inserting images

\title{Tutorial}
\author{Hector Sandoval \thanks{Gracias a Juan Valdes por sus explicaciones.}}
\date{\today}
% \date{July 2025}

\begin{document}

\maketitle
% \maketitle - generacion automatica de encabezado

\tableofcontents
\addcontentsline %for automatic table of contents

\begin{abstract}
    This is a short description of a scientific writing. Serves as a teaser.
\end{abstract}

\section{Introduction}

Just testing \textbf{bold}, \textit{italics}, \textsc{big letters} and \underline{underlined} formatting. 
%paragraphs are just marked by indentation
Double ENTER starts a new paragraph, but if I want to start a new line without making a new paragraph I just type double backlash \\ Really practical for keeping that low boilerplate.
% just use emp, if emphasis is what you want to achieve
Some of the greatest \emph{discoveries} in science 
were made by accident.

\textit{Some of the greatest \emph{discoveries} 
in science were made by accident.}

\textbf{Some of the greatest \emph{discoveries} 
in science were made by accident.}

\begin{figure}
    \centering
    \includegraphics[width=0.5\linewidth]{riverside.jpg}
    \caption{Wish I could chill in a place like that}
    \label{fig:enter-label}
\end{figure}

% \ref & \pageref to reference
% use to label as argument
% environments for typesetting = \begin{} \end{}
% figure, tabular, itemize, enumerate

\begin{itemize}
    \item example n1
    \item example n2
\end{itemize}

\begin{enumerate}
    \item item
    \item item
    \item item
\end{enumerate}

Why does everyone use the \(E = m c^2\) formula for display? I do not like it that much.
I feel like formulas like:
\begin{equation}% for numeration
    F = ma    
\end{equation}
have had more impact in my life, than that one. Even Hook's Law \[f = -kx\] has been more influential. But to speak truth, that formula has to be in a lot of technologies of this century, so... I should not be underestimating anything.

% _ for sub, ^ for upper
% int = integral, frac = fraction, sqrt = square root
% greek letters, write name ex.: \alpha (lower) \Alpha (upper)
% amsmath

% \section{} , \subsection{} , \subsubsection{} , etc.

\begin{center}%allign
\begin{tabular}{|c|c|}% l, left, or r, right. And | for vertical borders
    \hline %for horizontal borders
    cell1 & cell2 \\% \\ for next row
    \hline
    cell3 & cell4\\% \\ necessary for \hline to work
    \hline
\end{tabular}
\end{center}

\end{document}
